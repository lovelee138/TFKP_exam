\newpage
\section{Теорема о среднем и принцип максимума модуля. Принцип сохранения области.}

\textbf{Теорема о среднем:}
Пусть $f \in H(D), z_0 \in D, \partial U_{\rho}(z_0)\subset D$\\
Тогда $f(z_0) = \frac{1}{2\pi} \int\limits_{0}^{2\pi} f(z_0+\rho e^{it})dt$

\begin{proof}
    \ \\
    По интегральной формуле Коши:
    $$f(z_0)=\frac{1}{2\pi i}\int\limits_{\partial U_{\rho}(z_0)}\frac{f(z)}{z-z_0}dz = \frac{1}{2\pi i} \int\limits_{0}^{2\pi}\frac{f(z_0+\rho e^{it})}{\rho e^{it}}i\rho e^{it}dt=$$
    $$=\frac{1}{2\pi}\int\limits_0^{2\pi}f(z_0+\rho e^{it})dt$$
    Параметр $\partial U_{\rho}(z_0): \ z=z_0+\rho e^{it}: \ t\in[0; 2\pi], z'=i\rho e^{it}$
\end{proof}

\textbf{Принцип сохранения области:}\\[2mm]
Функция $f$ голоморфна в области $D$ и $f\neq const$
$$\Rightarrow$$
Образ $f(D)$ есть область

\begin{proof}
    \ \\
    Нужно показать, что множество $D^*$ связно и открыто.\\
    1) Пусть $w_1, w_2$ --- две произвольные точки $D^*$, а $z_1, z_2$ --- их прообразы.\\
    Так как множество $D$ линейно связно, то существует путь $\gamma: [\alpha; \beta]\rightarrow D$, связывающий точки $z_1, z_2$. В силу непрерывности функции $f$ образ $\gamma^*=f\circ \gamma$ будет путем, связывающим точки $w_1, w_2$. Таким образом $D^*$ --- линейно связно.\\[2mm]
    2) Пусть $w_0$ произвольная точка $D^*$ и $z_0$ --- один из ее прообразов в $D$. Так как $D$ открыто, то сущестует круг $\{|z-z_0|\leq r\}\subset D$. \\
    Будем уменьшать $r$ пока круг не перестанет содержать других точек, в которых функция $f$ равна $w_0$ (это возможно, т.к. она не постоянная).\\
    Обозначим $\gamma=\{|z-z_0|=r\}$ границу этого круга и
    $$\mu=\underset{x\in \gamma}{\min}|f(x)-w_0|, \ \mu>0$$
    Если бы $\mu =0$, то на $\gamma$ существовала бы точка, в которой функция $f$ равна $w_0$.\\[2mm]
    Теперь докажем, что $\{|w-w_0|<\mu\} \subset D^*$.\\
    Пусть $w_1$ --- произвольная точка этого круга, то есть $|w_1-w_0|<\mu$. Тогда:\\
    $f(z)-w_1=f(z)-w_0+(w_0-w_1)$.\\
    Так как на $\gamma$ в силу выбора $\mu$ имеем $|f(z)-w_0|\geq \mu$, то по теорему Руше функция $f(z)-w_1$ имеет внутри $\gamma$ столько же нулей, сколько и $d(z)-w_0$, то есть по крайней мере один нуль $z_0$, а значит функция $f$ внутри $\gamma$ принимает значение $w_1$, то есть $w_1 \in D^*$. В силу произвольности выбора $w_1$ весь круг лежит в $D^*$, а значит оно является открытым множеством.
\end{proof}


\textbf{Принцип максимума модуля:}\\[2mm]
Модуль голоморфной функции не может достигать строгого локального максимума внутри области.\\[2mm]
\begin{proof}
    \ \\
    Пусть функция достигает максимума в некоторой точке $z_0$.\\
    Воспользуемся принципом сохранения области. Если $f\neq const$, то она преобразует $z_0$ в точку $w_0$ области $D^*$. \\
    Существует круг $\{|w-w_0|<\mu\}\subset D^*$, а в нем найдется точка $w_1$ такакя, что  $|w_1|> |w_0|$. Значение $w_1$ принимается функцией $f$ в некоторой окрестности точки $z_0$, а это противоречит тому, что $|f|$ достигает максимума в этой точке.
\end{proof}
