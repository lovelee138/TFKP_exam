\newpage
\section{Три теоремы разложения. Доказать теоремы подобия и запаздывания.}

\textbf{Три теоремы разложения:}
\begin{enumerate}
    \item Если $F(p) = \sum_{n=0}^\infty\frac{a_n}{p^{n+1}}$ сходится при $|p|>R$, то:
    $$f(t)=\sum_{n=0}^\infty\frac{a_n}{n!}t^n$$
    \item Второй метод построения оригинала по изображению:\\
    Каждая рациональная функция $F(p)$, у которой степень числителя меньше степени знаменателя, является изображением:
    $$F(p)=\frac{R(p)}{Q(p)}, deg\,Q>deg\,R$$
    \item Пусть $f(t)\risingdotseq F(p); F(p)$ -- аналитическая функция при $Re\,p>\sigma_0$, а при $Re\,\leq \sigma_0$ имеет конечное число изолированных особых точек $p_1, ..., p_n$. Тогда:
    $$f(t)=\sum_{k=1}^{n}res(e^{pt}\cdot F(p))(p_k)$$
\end{enumerate}

\textbf{Теорема подобия:}\\
$\forall \lambda > 0: f(\lambda t) \risingdotseq \frac{1}{\lambda} F\left(\frac{b}{\lambda}\right)$

\begin{proof}
    \ \\
    $f(\lambda t)\risingdotseq \int\limits_{0}^{+\infty}f(\lambda t)e^{-pt}dt =
    \begin{vmatrix}
        \tau=\lambda t\\
        t=\frac{\tau}{\lambda}\\
        dt=\frac{1}{\lambda}d\tau\\
    \end{vmatrix}
    = \int\limits_0^{+\infty} f(\tau)e^{-\frac{p}{\lambda}\tau}\frac{1}{\lambda}d\tau = \frac{1}{\lambda}\int\limits_0^{+\infty}f(\tau)e^{-\frac{p}{\lambda}\tau}d\tau=\frac{1}{\lambda} F\left(\frac{b}{\lambda}\right)$
\end{proof}

\textbf{Теорема запаздывания:}\\
$\forall \tau > 0: f(t-\tau) \risingdotseq e^{-p\tau}\cdot F(p)$
    
\begin{proof}
    \ \\
    $f(t-\tau)\risingdotseq \int\limits_0^{+\infty} f(t-\tau)e^{-p\tau}dt = 
    \begin{vmatrix}
        t_1=t-\tau\\
        dt_1=dt\\
        t=t_1+\tau
    \end{vmatrix}
    = \int\limits_{-\tau}^{+\infty}f(t_1)\cdot e^{-pt_1-p\tau}dt_1=\int\limits_0^{+\infty}f(t_1)\cdot e^{-pt_1-p\tau}dt_1+\int\limits_{-\tau}^0 f(t_1)\cdot e^{-p1_1-p\tau}dt_1$\\
    Из определения оригинала: $f(t_1)=0$ при $t_1<0$. Тогда:\\
    $f(t-\tau)\risingdotseq e^{-p\tau}\int\limits_0^{+\infty} f(t_1)e^{pt_1}dt_1 = e^{-p\tau}\cdot F(p)$
\end{proof}
    
