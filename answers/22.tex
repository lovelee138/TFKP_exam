\newpage
\section{Определение преобразования Лапласа. Теорема о существовании изображения. Поведение изображения в бесконечно удаленной точке. Изображение элементарных функций (единичная функция Хевисайда, показательная и степенная функции). Теорема обращения.}


Пусть $f(t)$ --- функция, $t \in \mathbb{R}$\\
\textbf{Преобразование Лапласа} функции $f$ --- это $$F(p)=\int\limits_{0}^{+\infty}f(t)e^{-pt}dt, p \in \mathbb{C}$$
Обозначение: $F(p)\risingdotseq f(t), f(t)\fallingdotseq F(p)$\\[2mm]
Функцию $f(t)$ называют \textbf{оригиналом}
$$\Leftrightarrow$$
\begin{enumerate}
    \item $f(t)$ --- кусочно-непрерывная при $t\geq 0$
    \item $f()t)=0$ при $t<0$
    \item $\exists M > 0: \ \exists \alpha \in \mathbb{R}: \ |f(t)|\leq M e^{\alpha t}$
\end{enumerate}


\textbf{Теорема о существования изображения:}\\[2mm]
Пусть $f(t)$ --- оригинал\\
Тогда: 1) интеграл $F(p)=\int\limits_{0}^{+\infty} f(t)e^{-pt}dt$ сходится абсолютно в области $\{p\in\mathbb{C}: Re\,p>\alpha\}$\\
2) изображение $F(p)$ --- аналитическая функция в $U_{\alpha}$.

\begin{proof}
    \ \\
    $|f(t)|\leq M e^{\alpha t}$ --- условие 3) из определения оригинала\\
    Пусть $p=\delta +i s$, тогда:
    $$|f(t)e^{-pt}|=|f(t)|\cdot|e^{-\delta t}|\cdot |e^{-ist}|\leq M e^{\alpha t}e^{-\delta t}=Me^{(\alpha-\delta)t}$$
    $|F(p)|=\lim\limits_{b\to + \infty} \left| \int\limits_{0}^{b}f(t)e^{-pt}dt \right| \leq \text{по теореме об оценке}$\\
    $\leq \lim\limits_{b\to +\infty}\int\limits_{0}^{b}|f(t)e^{-pt}|dt \leq \lim\limits_{b\to \infty}Me^{(\alpha-\delta)t}dt = \lim\limits_{b\to +\infty}\frac{M}{\alpha-\delta}e^{(\alpha -\delta)t}|_0^b=$\\
    $=\lim\limits_{b\to +\infty}\frac{M}{\alpha-\delta}(e^{(\alpha-\delta)b}-1)=-\frac{M}{\alpha-\delta}=\frac{M}{\delta-\alpha}$, значит по признаку Вейерштрасса сходится равномерно.\\
    Значит $Re\,(p=\delta+i s)=\delta \Rightarrow Re \, pj>\alpha \Rightarrow F'(p)=\int\limits_0^{+\infty} (-t)f(t)e^{-pt}dt \Rightarrow F(p)$ --- аналитическая функция.  
\end{proof}

\textbf{Поведение изображения в бесконечности:}\\[2mm]
Если изображение $F(p)$ аналитическая функция в $\infty$, то $F(\infty)=0$.\\

\begin{proof}
    \ \\
    Из доказательства теоремы 1:
    $$|F(p)|\leq \frac{M}{\delta - \alpha},$$ где $\alpha=const, \, \delta=Re\,p$.\\
    Если $Re\,p\to\infty,$ то $\frac{M}{\delta-\alpha}\to 0 \Rightarrow |F(p)|\to 0,$ т.к. $F(p)\to F(\infty)$.
\end{proof}

\textbf{Изображение элементарных функций:}
\begin{enumerate}
    \item Функция Хевисайда $\nu$\\
    $\nu(t)=
    \begin{cases}
        1\text{, при }t\geq 0\\
        0\text{, при }t\geq 0
    \end{cases}$\\
    $\int\limits_{0}^{\infty}1e^{-pt}dt=-\frac{e^{-pt}}{p}|_0^{+\infty}$\\
    При $p=\delta +i s: \ |e^{-pt}|=|e^{-\delta t}\cdot e^{-i s t}|$\\
    Тогда $1\risingdotseq \frac{1}{p}$

    \item Показательная функция\\
    $e^{\alpha t}\risingdotseq \frac{1}{p-\alpha}$\\
    По теореме затухания $\forall \alpha \in \mathbb{C}: e^{\alpha t} f(t) \risingdotseq F(p-\alpha)$\\
    Так как $1\risingdotseq \frac{1}{p}$, то имеем:\\
    $e^{\alpha t}\cdot 1 \risingdotseq \frac{1}{p-a}$

    \item Степенная функция\\
    $t^n \risingdotseq \frac{n!}{p^{n+1}}$\\
    Докажем по индукции:\\
    По теореме об интегрировании оригинал $\int\limits_{0}^{t} f(\tau)d\tau \risingdotseq \frac{F(p)}{p}$\\
    При $n=0:$ выполняется. Пусть выполнено для $n=k$, тогда:\\
    $f(\tau)=\tau^k \risingdotseq \frac{k!}{p^k}$\\
    $t^{k+1}=(k+1)\cdot\int\limits_{0}^{t}\tau^k d\tau = (k+1)\cdot \frac{k!}{p^k \cdot p} = \frac{(k+1)!}{p^{k+1}}$\\[2mm]
\end{enumerate}

\textbf{Теорема обращения:}\\
Если $f(t)$ --- оригинал с постоянной $\alpha$;\\
\hspace*{1 cm} $F(p)\fallingdotseq f(t)$; $t$ -- точка, в которой $f(t)$ -- непрерывная.\\
То $f(t)=\frac{1}{2\pi i}\int\limits_{\gamma-i\infty}^{\gamma+i\infty} F(p)e^{pt}dp$ --- интеграл по прямой\\
\hspace*{1cm} $Re\, p=\gamma, \ \gamma > \alpha$\\
