\newpage
\section{Определить дробно-линейное отображение (ДЛО). Сформулировать и доказать конформность и групповое свойство ДЛО.}


\textbf{Дробно-линейные отображения} --- это функции вида: $$w=\frac{az+b}{cz+d}, \text{где } \ \begin{vmatrix}
    a&b\\
    c&d
\end{vmatrix} \neq 0, \ a,b,c,d\in \mathbb{C}$$

Доопределим функцию выше следующим образом:$$z=-\frac{d}{c}: \ w(-\frac{d}{c}) = \infty$$
$$z = \infty: \ w(\infty) = \left[\begin{gathered}\frac{a}{c}, \ c\neq 0\\
\infty, \ c=0
\end{gathered}
\right.$$
Тогда ДЛО: $w: \overline{\mathbb{C}}\to \overline{\mathbb{C}}$


\textbf{Конформность:}

Любое ДЛО --- конформное отображение $\overline{\mathbb{C}}\to \overline{\mathbb{C}}$

\begin{proof}
    \ \\
    1) Рассмотрим точки $z_0 \neq -\frac{d}{c}, \ \infty:$
    
    Тогда $w'=\frac{a(cz+d)-(az+b)c}{(cz+d)^2} = \frac{ad-bc}{(cz+d)^2} \neq 0$

    Значит функция $w$ голоморфна в точках $z_0$ и по теореме о конформности голоморфных отображений, она конформна в этих точках $z_0$.

    2) Рассмотрим точку $z_0=-\frac{d}{c}:$ $$z\overset{L}{\longrightarrow}w=\frac{az+b}{cz+d}\overset{L_0}{\longrightarrow} \xi=\frac{1}{w}$$ $$\mathbf{-\frac{d}{c}} \ \overset{L}{\longrightarrow} \ \mathbf{\infty} \ \overset{L_0}{\longrightarrow} \ \mathbf{0}$$
    Отображение $\xi=\frac{1}{w}$ сохраняет углы, то есть $L_0$ конформно.
    Рассмотрим $L_0\circ L$ в точке $z_0=-\frac{d}{c}:$
    $$(L_0\circ L)^{'}_{|z=-\frac{c}{d}} = \frac{cb-ad}{(az+b)^2}_{|_{z=-\frac{c}{d}}} \neq 0$$
    Значит отображение $L_0\circ L$ конформно в точке $z_0=-\frac{c}{d}$\\
    $L=L_0^{-1}\circ (L_0\circ L)$ конформно, так как композиция конформных и обратное к конформному отображению конформны.

    3) Рассмотрим точку $z_0=\infty$:
    $$\xi=\frac{1}{z} \ \overset{L_0}{\longleftarrow} \ z \ \overset{L}{\longrightarrow} \ w=\frac{az+b}{cz+d}$$
    $$\mathbf{0} \ \overset{L_0}{\longleftarrow} \ \mathbf{\infty} \ \overset{L}{\longrightarrow} \ \mathbf{\frac{a}{c}}$$
    Рассмотрим отображение $L\circ L_0^{-1}$:
    $$w=\frac{a\cdot \frac{1}{\xi}+b}{c\cdot \frac{1}{\xi}+d} = \frac{b\cdot \xi+a}{d\cdot \xi +c}$$
    $$w^{'}_{|_{0}} = \frac{dc-da}{(d\cdot\xi+c)^2}_{|_{\xi=0}}\neq 0$$
    Значит отображение $L\circ L_{0}^{-1}$ конформно в точке $\xi_0 = 0$.\\
    Тогда отображение $L=(L\circ L_{0}^{-1})\circ L_0$ конформно в точке $z_0=\infty$, так как композиция конформных и обратное к конформному отображению конформны. 
\end{proof}

\textbf{Групповое свойство ДЛО:}

Совокупность всех ДЛО $\Lambda$ образует некоммутативную группу $(\Lambda; \circ)$ относительно операции композиции.

\begin{proof}
    \ \\
    0) Замкнутость:\\
    $w=\frac{az+b}{cz+d}; \ \xi=\frac{a_1w+b_1}{c_1w+d_1}$\\[2mm]
    $\xi=\frac{a_1\cdot\frac{az+b}{cz+b}+b_1}{c_1\cdot\frac{az+b}{cz+b}+d_1} =\frac{a_1(az+b)+b_1(cz+d)}{c_1(az+b)+d_1(cz+d)}=$\\
    $= \frac{(a_a+b_1c)z+a_1b+b_1d}{(c_1a+d_1c)z+c_1b+d_1d}$\\[2mm]
    Определитель $\begin{vmatrix}
        a_1a+b_1c & a_1b+b_1d\\
        c_1a+d_1c & c_1b+d_1d
    \end{vmatrix}$ не равен 0, так как иначе композиция ДЛО была бы отображением в одну точку, но композиция биекций есть биекция.\\[2mm]
    1) Ассоциативность выполняется, так как композиция отображений ассоциативна\\[2mm]
    2) Существование единицы:\\
    $E: \ w=z, \ \begin{pmatrix} a=1&b=0\\c=0&d=1\end{pmatrix}, \ det = 1\neq0$\\[2mm]
    3) Существование обратного:\\
    Пусть $L: \ w=\frac{az+b}{cz+d}$ --- ДЛО.\\
    Построим обратное:
    $$w(cz+d)=az+b$$
    $$z(wc-a)=b-dw$$
    $$L^{-1}: \ z=\frac{b-dw}{wc-a} \text{ --- ДЛО, так как } \ \begin{vmatrix}-d&b\\c&-a\end{vmatrix}=ad-bc\neq0$$ 
    4) Некоммутативность:\\
    Приведем контрпример
    $$L_1: \ w=z+a, \ L_2: \ w=\frac{1}{z}$$
    $$L_1\circ L_2: \ z \overset{L_2}{\rightarrow} w=\frac{1}{z} \overset{L_1}{\rightarrow} w = \frac{1}{z}+a$$
    $$L_2 \circ L_1: \ z \overset{L_1}{\rightarrow} w = z+a \overset{L_2}{\rightarrow} w = \frac{1}{z+a}$$
    Получили, что $L_1\circ L_2 \neq L_2 \circ L_1$
\end{proof}
