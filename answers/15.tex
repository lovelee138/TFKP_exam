\newpage
\section{Изолированные особые точки голоморфных функций, их классификация и характеризация в терминах рядо Лорана. Поведение голоморфных функций в окрестности особых точек.}

\textbf{Точка }$\mathbf{z_0}$ --- \textbf{изолированная особая точка} функции $f$, если:
$$\exists r > 0: \ f\in H(\overset{\circ}{U_{r}}(z_0))$$

Разложение в ряд лорана в окрестности $z_0$:
$$f(z)=\sum_{n=-\infty}^{+\infty} c_n z^n\text{, обл. сход. содержит } U_r(z_0)$$

Главная часть: $\sum_{n=-\infty}^{-1} c_n z^n$

Правильная часть: $\sum_{n=0}^{+\infty} c_n z^n$

Классификация особой точки $a$ (характеризация):
\begin{enumerate}
    \item \textbf{Устранимая}, если существует конечный  предел функции в этой точке.
    \item \textbf{Полюс}, если предел функции равен бесконечности в этой точке.
    \item \textbf{Существенно особая}, если не существует предела функции в этой точке.
\end{enumerate}

\textbf{Характеризация устранимой особой точки:}\\
Если точка $a\in \mathbb{C}$ --- изолированная особая точка функции $f$, то следующие условия эквивалентны:
\begin{enumerate}
    \item $a$ --- устранимая особая точка
    \item Лорановское разложение функции $f$ в окрестности точки $a$ не содержит главной части. 
    \item Функция $f$ ограничена в окрестности точки $a$.
\end{enumerate}

\begin{proof}
    \ \\
    "$2)\to 1)$":\\
    Разложение в ряд Лорана в $\overset{\circ}{U_{\delta}}(a):$\\
    $f(z)=\sum_{n=0}^\infty c_n(z-a)^n \to c_0$ при $z\to a \Rightarrow 1)$\\[2mm]
    "$1) \to 3)$":\\
    $\lim\limits_{z\to a}f(z)=A\in\mathbb{C} \Rightarrow f$ -- ограниченная в некоторой окрестности точки $a$.\\[2mm]
    "$3) \to 2) $":\\
    $\exists \delta_1 >0 f\in H(\overset{\circ}{U}_{\delta_1}(a)) \Rightarrow$ по теореме Лорана\\
    $\Rightarrow f=\sum_{n=-\infty}^{+\infty}c_n(z-a)^n$ в $\overset{\circ}{U}_{\delta_1}(a).$\\
    Пусть $\delta_1 > 0; f$ ограниченна $M>0$ в $\overset{\circ}{U}_{\delta_1}(a), \rho \in (0; \delta_1)\Rightarrow$ по неравенству Коши $\Rightarrow \forall n\in\mathbb{Z} |c_n|\leq \frac{M}{\rho^n}$\\
    Если $n<0$, то $\frac{M}{\rho^n} \to 0$ при $\rho \to 0$. Но $|c_n|$ не зависит от выбора $\rho$. Поэтому $c_n = 0$ при $n<0 \Rightarrow 2)$.\\[2mm]
\end{proof}

\textbf{Характеризация полюса:}\\
Если точка $a\in \mathbb{C}$ --- изолированная особая точка функции $f$, то следующие условия эквивалентны:
\begin{enumerate}
    \item $a$ --- полюс
    \item Главная часть Лорановского разложения функции $f$ в окрестности точки $a$ содержит конечное (>0) число слагаемых:
    $$f(z)=\sum_{n=-N}^\infty c_n(z-a)^n,$$
    где $N>0, c_{-N}\neq 0$ 
    \item $f=\frac{1}{\varphi}$, где $\varphi$ --- голоморфная в точке $a$ и $\varphi(a) = 0, \ \varphi \not\equiv 0$.
\end{enumerate}

\begin{proof}
    \ \\
    "$2) \to 1)$":\\
    $f(z)=\sum_{n=-N}^\infty c_n(z-a)^n=\frac{1}{(z-a)^N}\sum_{n=-N}^\infty c_n (z-a)^{n+N},$\\
    где $\frac{1}{(z-a)^N}\to \infty$ при $z\to a$ и $\sum_{n=-N}^\infty c_n (z-a)^{n+N}$ --- степенной ряд $\Rightarrow c_{-N} \neq 0$.\\[2mm]
    "$1) \to 3)$":\\
    $\varphi(z)=
    \begin{cases}
        \frac{1}{f(z)}\text{, если }z\neq a\\
        0\text{, если }z=a
    \end{cases}
    \text{ --- голоморфнаяв точке }a$\\
    Функция $\frac{1}{f(z)}$ имеет устарнимую особую точку в $a \Rightarrow \frac{1}{f(z)}=\sum_{n=0}^\infty a_n(z-a)^n$\\
    $\lim\limits_{z\to a}\frac{1}{f(z)}=0\Rightarrow a_0 =0$\\[2mm]
    "$3) \to 2)$":
    Пусть $N$ -- такое, что $a_0 = 0 = a_1=...=a_{N-1}, \ a_N \neq 0$.   Тогда $\frac{1}{f(z)}=\sum_{n=N}^\infty a_n(z-a)^n=\varphi(z) \Rightarrow $
    $$\Rightarrow f(z)=\frac{1}{\sum_{n=N}^\infty a_n(z-a)^n}=\frac{1}{(z-a)^N}\cdot \frac{1}{\sum_{n=N}^\infty a_n (z-a)^{n-N}},$$
    где $\sum_{n=N}^\infty a_n (z-a)^{n-N}$ -- голоморфна в точке $a$, равна $a_N \neq 0$;\\
    $\frac{1}{\sum_{n=N}^\infty a_n (z-a)^{n-N}}$ -- голоморфна в окрестности точки $a$.\\
    $\Rightarrow f(z)=\frac{1}{(z-a)^N}\sum_{k=0}^\infty c_k (z-a)^k = \sum_{k=0}^\infty c_k (z-a)^{k-N}=$\\
    $=\sum_{n=-N}^\infty \tilde{c_n}(z-a)^n; \tilde{c_{-N}}=c_0\neq 0$\\[2mm]
\end{proof}

\textbf{Характеризация существенно особой точки:}\\
Изолированная особая точка $a$ функции $f$ является существенно особой
$$\Leftrightarrow$$
Главная часть Лорановского разложения функции $f$ в окрестности точки $a$ имеет бесконечное число слагаемых.\\
\begin{proof}
    \ \\
    Следует из характеризаций устранимой особой точки и вычета.\\[2mm]
\end{proof}


\textbf{Поведение голоморфных функций в окрестности особых точек.}
\textbf{Теорема Сохоцкого:}\\
Если $a$ -- существенная особая точка функции $f$, то $\forall A \in \overline{\mathbb{C}}$ $\exists \{z_n\}:$ 
$$z_n \rightarrow z_0 \text{ при } n \rightarrow \infty, f(z_n) \rightarrow A$$.\\[2mm]

\begin{proof}
    \ \\
    Пусть $A=\infty$. Так как $f$ не может бысть ограниченной в проколотой окрестности $\{0<|z-a|<r\}$ (из характеризации устранимой особой точки), то найдется в этой окрестности такая точка $z_1$, в которой $|f(z_1)|>1$. Точно так же в $\{0<|z-a|<\frac{|z_1-a|}{2}\}$ найдется точка $z_2$, в которой $|f(z_2)|>2$ и т.д., в $\{0<|z-a|<\frac{|z_1-a|}{n}\}$ найдется точка $z_n$, в которой $|f(z_n)|>n$. Очевидно, $z_n \to a$ и $\lim\limits_{n\to\infty} f(z_n)=\infty$.\\[2mm]
    Пусть теперь $A\neq \infty$. Либо точки, в которых $f$ равна $A$ имеют $a$ своей предельной точкой, и тогда из них можно выбрать последовательности $z_n \to a$, на которой $f(z_n)=A$, либо существует проколотая окрестность $\{0<|z-a|<r'\}$, в которой $f(z)\neq A$. В этой окрестности голоморфна функция $\varphi(z)=\frac{1}{f(z)-A}$, для которой $a$ также является существенно особой точкой (т.к. $f(z)=A+\frac{1}{\varphi(z)}$ и если бы $\varphi$ при $z\to a$ стремилась к конечному или бесконечному пределу, то $f$ -- также). По доказанному существует последовательность $z_n\to a$, по которой $\varphi(z_n)\to \infty$, но по этой последовательности:
    $$\lim\limits_{n\to \infty}f(z_n)=A+\lim\limits_{n\to\infty}\frac{1}{\varphi(z_n)}=A$$
\end{proof}
