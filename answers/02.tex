\newpage
\section{Геометрический смысл комплексной производной. Конформные отображения,
связь конформности и дифференцируемости, примеры.}

Пусть задана кривая $z = z(t) = x(t) + iy(t)$, имеющая касательную в точке $t_0$ c направляющим вектором $\xi = x'(t_0) + iy'(t_0)$.
Назовем $\xi$ касательным вектором в точке $t_0$ к кривой $z$.

\begin{samepage}
\textbf{Теорема (геометрический смысл комплексной производной):}
\begin{enumerate}
  \item Любая голоморфная в т. $z_0 = z(t_0)$ функция f определяет линейное отображение касательных касательных векторов $\eta = f'(z_0)\xi$, где $\eta$ -- образ касательного вектора $\xi$, являющийся касательным вектором к кривой $f(z)$ в точке $f(z_0)$.
  \item Это отображение касательных векторов состоит в растяжении с коэффициентом $|f'(z_0)|$ и повороте на угол $arg f'(z_0)$.
\end{enumerate}
\end{samepage}

$\square$
а) По правилу дифференцирования сложной функции (в смысле $\mathbb{R}^2$)
$$\eta = \frac{df(z(t))}{dt} (t_0) = f'(z(t_0))z'(t_0) = f'(z_0)\xi$$

б) $|\eta| = |f'(z_0)| \cdot |\eta|$ -- растяжение с коэффициентом $|f'(z_0)|$; 

$arg \eta = arg f'(z_0) + arg \xi \pm 2\pi$ -- поворот на угол $arg f'(z_0)$.

\hfill$\blacksquare$

Отображение $F$ называется \textbf{конформным} в точке $M_0(x_0,y_0)$ тогда и только тогда, когда касательное отображение в точке $M_0$ сохраняет углы.

Отображение $F$ называется \textbf{конформным} в области $U \subset \mathbb{R}^2$ тогда и только тогда, когда оно конформно в любой из точек $U$.

$U\subset \mathbb{C}$, $H(U)$ -- множество голоморфных в $U$ функций.

\textbf{Теорема (о связи конформности и диффиренцируемости):}\\
$U$ -- область в $\mathbb{C}$. Если $f\in H(U)$ и $\forall z\in U~~f'(z) \neq 0$, тогда $f$ -- конформное в $U$ отображение.

$\square$
Доказательство следует из предыдущей теоремы:\\
В каждой точке $z_0 \in U$ лин.отображение $f'(z_0)$ растигивает в $|f'(z_0)|\neq 0$ и поворачивает на угол $arg\,f'(z_0)\Rightarrow$ лин.отображение в $z_0$ сохраняет углы. 

\hfill$\blacksquare$

\textbf{Определение:} Угол между кривыми $\gamma_1$ и $\gamma_2$ в точке $\infty$ равнен углу между касательными к $\hat{\gamma}_1$ и $\hat{\gamma}_2$ в точке 0, где $\hat{\gamma}_1 = \frac{1}{\gamma_1}$ и $\hat{\gamma}_2 = \frac{1}{\gamma_2}$.

Отображение $F$ называется \textbf{конформным} в точке $\infty$ тогда и только тогда, угол между кривыми $\gamma_1$ и $\gamma_2$ в точке $\infty$ равен углу между кривыми $f(\gamma_1)$ и $f(\gamma_2)$ в точке $\infty$.
