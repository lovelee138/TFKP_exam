\newpage
\section{Степенные ряды в $\mathbb{C}$, их свойства. Голоморфность суммы степенного ряда.}

Ряд $\sum_{n=0}^\infty c_n(z-z_0)^n$ --- \textbf{степенной} ряд, $c_n \in \mathbb{C}$.\\[2mm]

\textbf{Свойства:}
\begin{enumerate}
    \item Теорема Абеля:\\
    Если степенной ряд $\sum_{n=0}^\infty c_n(z-z_0)^n$ сходится в точке $z_1$, то этот ряд сходится в круге $U=\{z\in \mathbb{C}: \, |z-z_0|<|z_1-z_0|\}$ и на любм компакте $K \subset U$ он сходится равномерно.
    \item Теорема Коши-Адамара:\\
    Пусть для ряда $A: \ \sum_{n=0}^\infty c_n(z-z_0)^n$ имеем $\overline{\lim\limits_{n\to\infty}\limits}\sqrt[n]{|c_n|} = \frac{1}{R}$, где $,\leq R \leq \infty$.\\
    Тогда в любой точке $z: \ |z-z_0|<R$ ряд сходится и в любой точке $z: \ |z-z_0|>R$ ряд расходится.
\end{enumerate}

\textbf{Голоморфность суммы степенного ряда:}\\[2mm]
Пусть в круге $U=\{z\in \mathbb{C}: \, |z-z_0|<R\}$  $S(z) = \sum_{n=0}^\infty c_n (z-z_0)^n$.\\
Тогда $S \in H(U_R(z_0))$ и $S'(z)=\sum_{n=1}^\infty n\cdot c_n(z-z_0)^{n-1} \ (*)$

\begin{proof}
    \ \\
    $r: \ 0<r<R$ -- произвольные.\\
    Пусть $z_1 \in U_R(z_0): \ |z-z_0| > r$\\
    $\forall z \in U_r(z_0) = \{z: |z-z_0|<r\}:$\\
    $|n\cdot C_n(z-z_0)^{n-1}| = n\left|C_n\frac{(z-z_0)^{n-1}}{(z_1-z_0)^n}\right|\cdot |(z_1-z_0)^n|=n\frac{1}{|z_1-z_0|}\cdot |C_n (z_1-z_0)^n|\cdot \left|\frac{z-z_0}{z_1-z_0}\right|^{n-1}\leq n\frac{M}{|z_1-z_0|}\rho^{n-1}$,\\
    где $M > |C_n (z_1-z_0)^n|, \ \rho = \left|\frac{z-z_0}{z_1-z_0}\right|$\\
    То есть ряд $\sum_{n=1}^\infty n\frac{M}{|z_1-z_0|}\rho^{n-1}=\frac{M}{|z_1-z_0|}\sum_{n=1}^\infty n\rho^{n-1}$ -- мажорирующий для ряда $(*)$.\\
    Ряд $\sum_{n=1}^\infty n\rho^{n-1}$ сходится при $\rho \in (0; 1)$ как ряд из производных ряда $\sum_{n=1}^\infty\rho^n$. Тогда по признаку Вейерштрасса ряд $(*)$ сходится равномерно и абсолютно в $U_r(z_0)$.

    Для любой замкнутой кривой $\gamma \subset U_r(z_0)$ по теореме Коши:\\
    $\oint\limits_{\gamma}\left( \sum_{n=1}^\infty n C_n (z-z_0)^{n-1} \right)dz=\sum_{n=1}^\infty C_n \oint\limits_{\gamma}(z-z_0)^{n-1}dz=0$\\
    Значит функция $g(z)=\sum_{n=1}^\infty n\cdot C_n(z-z_0)^{n-1}$ имеет первообразную в $U_r(z_0)$, которая равна:\\
    $\int\limits_{z_0}^z g(\xi)d\xi = \int\limits_{z_0}^z\sum_{n=1}^{\infty} n\cdot C_n (\xi-z_0)^{n-1}d\xi=\sum_{n=1}^{\infty} n C_n \frac{(z-z_0)^n}{n} = S(z)-S(z_0) = S(z)-C_0$.\\
    Следовательно $S \in H(U_r(z_0)) \forall r \in (0; R)$.\\
    Поэтому $S \in H(U_R(z_0))$ и $S'=g$.
\end{proof}

\textbf{Следствия} из этой теоремы:
\begin{enumerate}
    \item Производная функции $f\in H(d)$ голоморфна в $D$
    \item 
\end{enumerate}
