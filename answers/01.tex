\newpage 
\section{Непрерывность и дифференцируемость функций коплексного переменного, их связь. Теорема Коши-Римана. Голоморфные функции.}
\ \\
ФКП $f: \, G\subset \overline{\mathbb{C}} \rightarrow \overline{\mathbb{C}}$ \textbf{непрерывна} в точке $z_0$, если:
$$\lim\limits_{z\to z_0}f(z)=f(z_0)$$
ФКП $f(z) \,\mathbf{\mathbb{C}}$\textbf{--дифференцируема в точке $z_0$} $\Leftrightarrow$ 
\begin{enumerate}
    \item $f$ определена в окрестности точки $z_0$;
    \item $f(z_0 + \Delta z) - f(z_0) = A\Delta z + \alpha(\Delta z)\Delta z$,
\end{enumerate}
где $A \in \mathbb{C}$, $\alpha(\Delta z) \rightarrow 0$ при $\Delta z \rightarrow 0$\\[2mm]
Предел $\lim_{\Delta z \rightarrow 0} \frac{f(z_0 + \Delta z) - f(z_0)}{\Delta z}$ называют \textbf{производной ФКП $f(z)$ в точке $z_0$} и обозначают $f'(z_0)$.\\[2mm]
\textbf{Теорема (1-ый критерий $\mathbf{\mathbb{C}}$--дифференцируемости):}
\begin{center}
    ФКП $f(z)$ дифференцируема в точке $z_0$\\
    $\Leftrightarrow$\\
    $\exists$ производная $f'(z_0)$ функции $f$ в точке $z_0$, при этом $f'(z_0) = A$.
\end{center}
\begin{proof}
\ \\
''$\Rightarrow$''\\
$f'(z_0) = \lim_{\Delta z \rightarrow 0} \frac{f(z_0 + \Delta z) - f(z_0)}{\Delta z} = \lim_{\Delta z \rightarrow 0} \frac{A\Delta z + \alpha(\Delta z)\Delta z}{\Delta z} = \\
= \lim_{\Delta z \rightarrow 0} (A + \alpha(\Delta z)) = A$ $(\alpha(\Delta z) \rightarrow 0$ при $\Delta z \rightarrow 0) \Rightarrow$\\
$\Rightarrow \exists f'(z_0) = A$.\\
''$\Leftarrow$''\\
$f'(z_0) = \lim_{\Delta z \rightarrow 0} \frac{f(z_0 + \Delta z) - f(z_0)}{\Delta z}$,\\
$\alpha(\Delta z) = \frac{f(z_0 + \Delta z) - f(z_0)}{\Delta z} - f'(z_0) \rightarrow 0$ при $\Delta z \to 0 \Rightarrow$\\
$\Rightarrow f(z_0 + \Delta z) - f(z_0) = A\Delta z + \alpha(\Delta z)\Delta z$.
\end{proof}
Функция $w = f(z)$ называется \textbf{голоморфной (аналитической)} в точке $z_0 \in \mathbb{C} \Leftrightarrow f$ -- $\mathbb{C}$ -- дифференцируема в окрестности точки $z_0$.\\[2mm]
\textbf{Теорема (об условиях Коши-Римана):}\\[2mm]
Функция $f(z) = u(x, y) + iv(x, y)$, где $z = x + iy$, $\mathbb{C}$ -- дифференцируема в точке $z_0 = x_0 + iy_0$ тогда и только тогда, когда:
\begin{enumerate}
    \item Функции $u(x, y)$ и $v(x, y)$ $\mathbb{R}^2$ -- дифференцируемы в точке $M_0(x_0, y_0)$;
    \item Выполняются условия (уравнения) Коши-Римана:\\
    $\begin{cases}
        \dfrac{\partial u}{\partial x} (M_0) = \frac{\partial v}{\partial y} (M_0)\\
        \dfrac{\partial u}{\partial y} (M_0) = -\frac{\partial v}{\partial x} (M_0)
    \end{cases}$
\end{enumerate}
При этом $f'(z_0) = \frac{\partial u}{\partial x} (M_0) + i\frac{\partial v}{\partial x} (M_0) = \frac{\partial v}{\partial y} (M_0) - i\frac{\partial u}{\partial y} (M_0)$.
\begin{proof}
\ \\
''$\Rightarrow$''\\
$\Delta f(z_0, \Delta z) = A\Delta z + \gamma(\Delta z)\Delta z$, но при этом:\\
$\Delta f(z_0, \Delta z) = \Delta u(x_0, y_0, \Delta x, \Delta y) + i\Delta v(x_0, y_0, \Delta x, \Delta y) =$\\[2mm]
$ = \alpha\Delta x - \beta\Delta y + i(\alpha\Delta y + \beta\Delta x) + \gamma(\Delta z)\Delta z \Rightarrow$\\[2mm]
$\Rightarrow
\begin{cases}
    \Delta u = \alpha\Delta x - \beta\Delta y + Re(\gamma(\Delta z)\Delta z)\\
    \Delta v = \alpha\Delta y + \beta\Delta x + Im(\gamma(\Delta z)\Delta z),
\end{cases}$\\
где $\frac{\gamma(\Delta z)\Delta z}{|\Delta z|} \rightarrow 0$ при $\Delta z \rightarrow 0 \ ((\Delta x, \Delta y) \rightarrow 0)$, так как $\gamma(\Delta z) \rightarrow 0$, а $\frac{\Delta z}{|\Delta z|}$ -- ограничена при $\Delta z \rightarrow 0 \Rightarrow$\\
$\begin{cases}
    Re(\gamma(\Delta z)\Delta z) = o(|\Delta z|)\\
    Im(\gamma(\Delta z)\Delta z) = o(|\Delta z|)
\end{cases}$
$\Rightarrow$ 1);\\
$\begin{cases}
    \Delta u = u'_x\Delta x + u'_y\Delta y + o(|\Delta z|)\\
    \Delta v = v'_x\Delta x + v'_y\Delta y + o(|\Delta z|)
\end{cases} \Rightarrow$\\
$\Rightarrow
\begin{cases}
    u'_x = \alpha = v'_y\\
    u'_y = -\beta = -v'_x
\end{cases} \Rightarrow 2)$\\
Тогда $f'_z = \alpha + i\beta \Rightarrow$
$\Rightarrow f'(z_0) = \frac{\partial u}{\partial x} (M_0) + i\frac{\partial v}{\partial x} (M_0) = \frac{\partial v}{\partial y} (M_0) - i\frac{\partial u}{\partial y} (M_0)$\\[2mm]
``$\Leftarrow:$''
Аналогично
\end{proof}
