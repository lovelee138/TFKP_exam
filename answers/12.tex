\newpage
\section{Бесконечная дифференцируемость голоморфных функций. Единственность разложения в степенной ряд. Теорема Морера. Эквивалентность голоморфности в смысле Римана, Коши и Вейерштрасса.}


\textbf{Голоморфность суммы степенного ряда:}\\[2mm]
Пусть в круге $U=\{z\in \mathbb{C}: \, |z-z_0|<R\}$  $S(z) = \sum_{n=0}^\infty c_n (z-z_0)^n$.\\
Тогда $S \in H(U_R(z_0))$ и $S'(z)=\sum_{n=1}^\infty n\cdot c_n(z-z_0)^{n-1} \ (*)$

\begin{proof}
    \ \\
    $r: \ 0<r<R$ -- произвольные.\\
    Пусть $z_1 \in U_R(z_0): \ |z-z_0| > r$\\
    $\forall z \in U_r(z_0) = \{z: |z-z_0|<r\}:$\\
    $|n\cdot C_n(z-z_0)^{n-1}| = n\left|C_n\frac{(z-z_0)^{n-1}}{(z_1-z_0)^n}\right|\cdot |(z_1-z_0)^n|=n\frac{1}{|z_1-z_0|}\cdot |C_n (z_1-z_0)^n|\cdot \left|\frac{z-z_0}{z_1-z_0}\right|^{n-1}\leq n\frac{M}{|z_1-z_0|}\rho^{n-1}$,\\
    где $M > |C_n (z_1-z_0)^n|, \ \rho = \left|\frac{z-z_0}{z_1-z_0}\right|$\\
    То есть ряд $\sum_{n=1}^\infty n\frac{M}{|z_1-z_0|}\rho^{n-1}=\frac{M}{|z_1-z_0|}\sum_{n=1}^\infty n\rho^{n-1}$ -- мажорирующий для ряда $(*)$.\\
    Ряд $\sum_{n=1}^\infty n\rho^{n-1}$ сходится при $\rho \in (0; 1)$ как ряд из производных ряда $\sum_{n=1}^\infty\rho^n$. Тогда по признаку Вейерштрасса ряд $(*)$ сходится равномерно и абсолютно в $U_r(z_0)$.

    Для любой замкнутой кривой $\gamma \subset U_r(z_0)$ по теореме Коши:\\
    $\oint\limits_{\gamma}\left( \sum_{n=1}^\infty n C_n (z-z_0)^{n-1} \right)dz=\sum_{n=1}^\infty C_n \oint\limits_{\gamma}(z-z_0)^{n-1}dz=0$\\
    Значит функция $g(z)=\sum_{n=1}^\infty n\cdot C_n(z-z_0)^{n-1}$ имеет первообразную в $U_r(z_0)$, которая равна:\\
    $\int\limits_{z_0}^z g(\xi)d\xi = \int\limits_{z_0}^z\sum_{n=1}^{\infty} n\cdot C_n (\xi-z_0)^{n-1}d\xi=\sum_{n=1}^{\infty} n C_n \frac{(z-z_0)^n}{n} = S(z)-S(z_0) = S(z)-C_0$.\\
    Следовательно $S \in H(U_r(z_0)) \forall r \in (0; R)$.\\
    Поэтому $S \in H(U_R(z_0))$ и $S'=g$.
\end{proof}

\textbf{Следствие} из этой теоремы:
Производная функции $f\in H(D)$ голоморфна в $D$
\begin{proof}
    \ \\
    $z_0 \in D$ -- произвольная точка множества $D \ \Rightarrow$ $z_0$ -- внутренняя точка $D$, 
    так как $D$ -- открытое множество $\Rightarrow$ \\[2mm]
    $\exists R > 0: U_R(z_0) \subset D \Rightarrow$ \\[2mm]
    $(\text{теорема о разложении голоморфной функции в ряд})$ \\[2mm] 
    $f(z) = \sum_{n=0}^{\infty}c_n (z - z_0)^n \Rightarrow$ (голом. степенного ряда)\\[2mm]
    $\Rightarrow \ f'(z)$ -- сумма степенного ряда, значит голоморфна 
\end{proof}

Из следствия 1 теоремы о разложении функции в степенной 
ряд следует бесконечная дифференцируемость голоморфных функций.


\textbf{Теорема о единственности разложения в степенной ряд:}\\[2mm]
Если в $U_R(z_0) \ f(z) = \sum_{n=0}^{\infty}c_n (z-z_0)^n$, то $c_n = \frac{f^{(n)}(z_0)}{n!}$


\begin{proof}
	\ \\
	\begin{equation*}
	\begin{gathered}
		f(z_0) = c_0 \\
		f'(z_0) = \sum_{n=1}^{\infty}nc_n(z-z_0)^{n-1} = c_1 \\
		\dots \\
		f^{(k)} = \sum_{n=k}^{\infty} n(n-1)\dots(n-k+1)c_n(z-z_0)^{n-k} = \\ 
		= k(k-1) \cdot \dots \cdot 1 = k!c_k \Rightarrow \\ 
		\Rightarrow c_k = \frac{f^{(k)}(z_0)}{k!}
	\end{gathered}
	\end{equation*}
\end{proof}


\textbf{Теорема Морера:}\\[2mm]
Пусть $D \subset \mathbb{C}$ -- область, $f \in C(D)$ и $\int \limits_{\partial \Delta} = 0$
для произвольного треугольника $\Delta$ при $\Delta \cup \partial \Delta \subset D$. \\[2mm]
Тогда $f \in H(D)$


\begin{proof}
	\ \\
	Пусть $a \in D$ -- призвольная точка. \\[2mm]
	Так как $D$ -- открытое, то $\exists r: U_r(a) \subset D$ \\[2mm]
	Рассмотрим функцию $F = \int \limits_{[a, z]} f(\xi)d\xi$, $z \in U_r(z_0)$ \\[2mm]
	Аналогично доказательству теоремы о первообразной $F \in H(D)$ и $F'(z) = f(z)$ \\[2mm]
	Из голоморфности производной голоморфной функции следует утверждение теоремы
\end{proof}


\textbf{Теорема об эквивалетности трех определений голоморфности:}
3 следующих утверждения эквивалентны:
R) функция $f$ в некоторой окрестности $U(a)$ имеет комплексную производную (Риман) \\[2mm]
C) $f \in C(U(a))$ и $\int \limits_{\partial \Delta}f(z)dz = 0$ для любого треугольника $\Delta \subset U(a)$ (Коши) 
W) функция $f$ разложима в степенной ряд в окрестности точки $a$ по $(z-a)$ (Вейерштрасс)


\begin{proof}
	\ \\
	$R) \Rightarrow C)$ -- из теоремы Коши \\[2mm]
	$R) \Rightarrow W)$ -- по теореме о разложении голоморфной функции степенной ряд \\[2mm]
	$W) \Rightarrow R)$ -- теорема о голоморфности суммы степенного ряда \\[2mm]
	$C) \Rightarrow R)$ -- по теореме Морера
\end{proof}
