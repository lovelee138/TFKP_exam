\newpage
\section{Нули голоморфной функции, их свойства. Теорема единственности. Вычисление порядка нуля.}

\textbf{Определение:}\\[2mm]
Нулем функции $f$ называется точка $a \in \mathbb{C}: f(a) = 0$


\textbf{Теорема:}\\[2mm]
Если $f(a) = 0, \ f$ голофорфна в точке $a$, и $f \equiv 0$ в какой то окрестности точки $a$,
то $\exists n \in \mathbb{N}: f(z) = (z - a)^n\varphi(z)$, где $\varphi(z) \neq 0$ и $\varphi$ голоморфна в точке $a$


\begin{proof}
    \ \\
    По теореме о разложении голоморфной функции в степенной ряд:\\[2mm]
    $f(z) = \sum_{n=0}^{\infty} c_n (z-a)^n$ в некоторой окрестности точки $a$\\[2mm]
    $f(a) = c_0 = 0, \ \exists n \in \mathbb{N} \ c_n \neq 0$ (иначе $f(z) \equiv 0$)\\[2mm]
    Пусть $n \in \mathbb{N}$ -- такое, что $c_0 = c_1 = \dots = c_{n-1} = 0$. Тогда \\[2mm]
    $f(z) = \sum_{n=0}^{\infty}c_n (z-a)^n = \sum_{k=n}^{\infty}c_k(z-a)^k = (z-a)^k\sum_{k=n}^{\infty}c_k(z-a)^{k-n}$\\[2mm]
    Функция $\varphi = \sum_{n=k}^{\infty}c_k(z-a)^{k-n}$ и есть искомая функция (голоморфна и не ноль в $a$)
\end{proof}


\textbf{Следствие:}\\[2mm]
Если $f(a) = 0$, $f$ голоморфна в точке $a$, то
существует выколотая окрестность точки $a$, где функция не имеет нулей,
то есть ее нули -- изолированные точки


\textbf{Теорема о порядке нуля голомофрной функции:}\\[2mm]
Порядок нуля $a \in \mathbb{C}$ голоморфной функции $f$ 
совпадает с $n$ в формуле $f(z) = (z-a)^n\varphi(z)$


\begin{proof}
    \ \\
    В ходе доказательства теоремы o представлении
    голоморфной функции имеющей нуль было показано,\\[2mm] 
    что $c_0 = \dots c_{n-1}$. Из того что $c_k = f^{(k)}(a)$ 
    следует доказательство теоремы
\end{proof}


\textbf{Теорема единственности:}\\[2mm]
Если $D$ -- область в $\mathbb{C}$; $f_1, f_2 \in H(D), \ \forall z \in \mathcal{E} \subset D: f_1 = f_2$,\\[2mm]
$a$ -- предельная точка множества $\mathcal{E}$ и $a \in D$, то $f_1 = f_2$ на всем $D$


\begin{proof}
    \ \\
    $f = f_1 - f_2 \in H(D), \ \mathcal{R} = \{z \in D: f_1 = f_2\}$. \\[2mm] 
    Тогда $a$ -- предельная точка множества $\mathcal{R}$ \\[2mm]
    Тогда есть последовательность $\{z_n\}, \ z_n \rightarrow a$ при $n \rightarrow \infty$. \\[2mm]
    Из непрерывности $f$ следует что $\lim \limits_{z_n}f(z_n) = 0$, а \\[2mm]
    Из того что $a$ -- предельная точка множества $\mathcal{R}$ и следствия теоремы следует, что \\[2mm] 
    $f \equiv 0 \Rightarrow f_1 = f_2$ в некоторой окрестности точки $a$ \\[2mm]
    Из того что $a$ -- произвольная предельная точка имеем, что $\mathcal{R}$ -- замкнутное подмножество $D$ \\[2mm]
    Из связности $D$ следует, что $Int \mathcal{R} = D$
\end{proof}
